\documentclass[handout]{beamer}
\usetheme{boxes}

\usecolortheme{orchid}
\definecolor{teal}{HTML}{009999}
\definecolor{lightteal}{HTML}{00CCCC}
\definecolor{purple}{HTML}{990099}
\definecolor{lightpurple}{HTML}{CC00CC}
\definecolor{darkgrey}{HTML}{666666}
\definecolor{lightgrey}{HTML}{AAAAAA}
\definecolor{darkred}{HTML}{660000}
\definecolor{darkgreen}{HTML}{006600}
\definecolor{darkblue}{HTML}{000066}
\definecolor{lightred}{HTML}{AA0000}
\definecolor{lightgreen}{HTML}{00AA00}
\definecolor{lightblue}{HTML}{0000AA}
\setbeamercolor{title}{fg=darkblue}
\setbeamercolor{frametitle}{fg=darkblue}
\setbeamercolor{itemize item}{fg=teal}
\setbeamercolor{itemize subitem}{fg=lightteal}

\usepackage{fancybox}
\usepackage{adjustbox}
\usepackage{mathptmx}
\usepackage{amssymb}
\usepackage{listings}
\usepackage{amsmath}

% For math brackets
\usepackage{amsmath}

% gets rid of bottom navigation bars
\setbeamertemplate{footline}[frame number]{}

% gets rid of bottom navigation symbols
\setbeamertemplate{navigation symbols}{}

% uses number labels in bibliography
\setbeamertemplate{bibliography item}{\insertbiblabel}

% my style for emphasising stuff in colours
\newcommand{\strong}[1]{\textbf{\color{teal} #1}}
\newcommand{\stronger}[1]{\textbf{\color{purple} #1}}

% title, etc
\title{Time Series and Survival Data Mining\\{\small (lecture 8)}}
\subtitle{AGR9013 -- Introduction to Data Mining\\
{\tiny (version 1)}}
\author{Dr Ionut Moraru}
\institute{University of Lincoln, School of Agri-Foods Technology and Manufacturing}
\logo{\includegraphics[width=1cm]{images/uol-logo.png}}
\date{21-November-2023}

\addtobeamertemplate{navigation symbols}{}{\hspace{1em}    \usebeamerfont{footline}%
    \insertframenumber / \inserttotalframenumber }

% Remove navigation symbols
\beamertemplatenavigationsymbolsempty

\begin{document}

\lstset{ %
  backgroundcolor=\color{white},   % choose the background color; you must add \usepackage{color} or \usepackage{xcolor}
  basicstyle=\footnotesize,        % the size of the fonts that are used for the code
  belowcaptionskip=0pt,            % reduce space after the listings caption
  belowskip=0pt,                   % how much vertical space to skip after a listing
  breakatwhitespace=false,         % sets if automatic breaks should only happen at whitespace
  breaklines=true,                 % sets automatic line breaking
  caption={},                      % default to no caption
  captionpos=b,                    % sets the caption-position to bottom
  commentstyle=\color{lightgreen}, % comment style
  deletekeywords={...},            % if you want to delete keywords from the given language
  escapeinside={\%*}{*)},          % if you want to add LaTeX within your code
  extendedchars=true,              % lets you use non-ASCII characters; for 8-bits encodings only, does not work with UTF-8
  frame=single,                    % adds a frame around the code
  keepspaces=true,                 % keeps spaces in text, useful for keeping indentation of code (possibly needs columns=flexible)
  keywordstyle=\color{blue},       % keyword style
  language=Python,                 % the language of the code
  mathescape=true,                 % allow math mode within code
  morekeywords={then,...},         % if you want to add more keywords to the set
  numbers=left,                    % where to put the line-numbers; possible values are (none, left, right)
  numbersep=5pt,                   % how far the line-numbers are from the code
  numberstyle=\tiny\sf\color{lightgrey}, % the style that is used for the line-numbers
  rulecolor=\color{black},         % if not set, the frame-color may be changed on line-breaks within not-black text (e.g. comments (green here))
  showspaces=false,                % show spaces everywhere adding particular underscores; it overrides 'showstringspaces'
  showstringspaces=false,          % underline spaces within strings only
  showtabs=false,                  % show tabs within strings adding particular underscores
  stepnumber=1,                    % the step between two line-numbers. If it's 1, each line will be numbered
  stringstyle=\color{darkred},     % string literal style
  tabsize=3,                       % sets default tabsize to 3 spaces
  title=\lstname                   % show the filename of files included with \lstinputlisting; also try caption instead of title
}

%--------------------------------------------------------------------------------
\frame{\titlepage}


%--------------------------------------------------------------------------------
\section*{OUTLINE}
%--------------------------------------------------------------------------------
\begin{frame}{Outline}
\begin{itemize}
\item[] Part I. Time Series Data Mining
	\begin{itemize}
    \item[I.1.] Time Series Data
    \item[I.2.] Time Series Analysis
    \item[I.3.] Time Series Mining
	\end{itemize}
\item[] Part II. Survival Data Mining
	\begin{itemize}
    \item[II.1] Survival Analysis
    \item[II.2] Survival Curves
    \item[II.3] Hazard
    \item[II.4] Censoring
    \item[II.5] Forecasting
    \item[II.6] Survival Trees
	\end{itemize}
\vspace*{0.3cm}
\item To Do
\end{itemize}
\end{frame}
%--------------------------------------------------------------------------------


%--------------------------------------------------------------------------------
\section{PART I}
%--------------------------------------------------------------------------------
\begin{frame}{Part I: Time Series Data Mining}
\begin{itemize}
\item[I.1.] Time Series Data
\item[I.2.] Time Series Analysis
\item[I.3.] Time Series Mining
\end{itemize}
\end{frame}
%- - - - - - - - - - - - - - - - - - - - - - - - - - - - - - - - - - - - - - - - 
%- - - - - - - - - - - - - - - - - - - - - - - - - - - - - - - - - - - - - - - - 
\begin{frame}{I.1. Time Series Data (p1)}
\begin{center}
\includegraphics[width=\textwidth]{images/dm_process_ts.pdf}
\end{center}
\end{frame}
%- - - - - - - - - - - - - - - - - - - - - - - - - - - - - - - - - - - - - - - - 
\begin{frame}{I.1. Time Series Data (p2)}
\begin{itemize}
\item \stronger{Time Series} data refers to values over time repeatedly
\item Examples of time series data include: daily rainfall, stock market prices and exchange rates
\item Time series analysis concerns developing models to describe time series data
\item Time series data can be exploited in data mining too
\end{itemize}
\end{frame}
%- - - - - - - - - - - - - - - - - - - - - - - - - - - - - - - - - - - - - - - - 
\begin{frame}{I.1. Time Series Data (p3)}
\begin{itemize}
\item[] \strong{The time series aspect can occur in multiple ways}
\item[]
\item As the target to model: every instance is a time series
\item[]
\item As historical data on the customer or item of interest: for every instance (customer), there is a time series
\item[]
\item To potentially enhance existing customer level data: for every instance in a data set, there are separate time series
\end{itemize}
\end{frame}
%- - - - - - - - - - - - - - - - - - - - - - - - - - - - - - - - - - - - - - - - 
\begin{frame}{I.1. Time Series Data (p4)}
\begin{itemize}
\item Data for the quarterly earnings per share from Johnson \& Johnson: 
\end{itemize}
% latex table generated in R 3.3.2 by xtable 1.8-2 package
% Wed Mar  8 13:30:23 2017
\begin{center}
{\small
\begin{tabular}{rrrrr}
  \hline
 & Q1 & Q2 & Q3 & Q4 \\ 
  \hline
1960 & 0.71 & 0.63 & 0.85 & 0.44 \\ 
  1961 & 0.61 & 0.69 & 0.92 & 0.55 \\ 
  1962 & 0.72 & 0.77 & 0.92 & 0.60 \\ 
  1963 & 0.83 & 0.80 & 1.00 & 0.77 \\ 
  1964 & 0.92 & 1.00 & 1.24 & 1.00 \\ 
  1965 & 1.16 & 1.30 & 1.45 & 1.25 \\ 
  1966 & 1.26 & 1.38 & 1.86 & 1.56 \\ 
$\dots$ &  $\dots$ & $\dots$ & $\dots$ & $\dots$ \\
  1974 & 6.03 & 6.39 & 6.93 & 5.85 \\ 
  1975 & 6.93 & 7.74 & 7.83 & 6.12 \\ 
  1976 & 7.74 & 8.91 & 8.28 & 6.84 \\ 
  1977 & 9.54 & 10.26 & 9.54 & 8.73 \\ 
  1978 & 11.88 & 12.06 & 12.15 & 8.91 \\ 
  1979 & 14.04 & 12.96 & 14.85 & 9.99 \\ 
  1980 & 16.20 & 14.67 & 16.02 & 11.61 \\ 
   \hline
\end{tabular}
}
\end{center}
\end{frame}
%- - - - - - - - - - - - - - - - - - - - - - - - - - - - - - - - - - - - - - - - 
\begin{frame}{I.1. Time Series Data (p5)}
\begin{center}
\includegraphics[width=0.8\textwidth]{images/ts-1.png}
\end{center}
\end{frame}
%- - - - - - - - - - - - - - - - - - - - - - - - - - - - - - - - - - - - - - - - 
\begin{frame}{II.. Time Series Data (p6)}
\begin{itemize}
\item A \stronger{Time Series} is a sequence of random variables, $x_1, x_2, \ldots, x_t$, where $x_1$ denotes the value at time $1$ or the first time point
\item In the previous slide, $x_i$ represented the quarterly earnings at Johnson \& Johnson
\item[]
\item When plotting time series data, usually the time axis $t$ is plotted on the horizontal axis and the values for $x_i$ on the vertical axis
\item Successive observations are not independent, past observations influence future ones
\end{itemize}
\end{frame}
%- - - - - - - - - - - - - - - - - - - - - - - - - - - - - - - - - - - - - - - - 
%- - - - - - - - - - - - - - - - - - - - - - - - - - - - - - - - - - - - - - - - 
\begin{frame}{I.2. Time Series Analysis (p1)}
\begin{itemize}
\item[] Objectives of \stronger{Time Series Analysis}:
\item[]
\item Compact description of data
\item Interpretation and understanding, e.g., seasonal factors or covariates
\item Forecasting or prediction
\item Hypothesis testing
\end{itemize}
\end{frame}
%- - - - - - - - - - - - - - - - - - - - - - - - - - - - - - - - - - - - - - - - 
\begin{frame}{I.2 Time Series Analysis (p2)}
\begin{itemize}
\item[] Classic methods of time series analysis decompose a series into:
\item[]
\item Trend ($T_t$)
\item Cycle ($C_t$)
\item Seasonality ($S_t$)
\item Error ($\epsilon_t$)
\item[]
\item Typically the assumption is that these components are additive:
\[
	x_t = T_t+C_t+S_t+ \epsilon_t
\]
\end{itemize}
\end{frame}
%- - - - - - - - - - - - - - - - - - - - - - - - - - - - - - - - - - - - - - - - 
\begin{frame}{I.2 Time Series Analysis (p3)}
\begin{center}
\includegraphics[width=0.8\textwidth]{images/ts-2.png}
\end{center}
\end{frame}
%- - - - - - - - - - - - - - - - - - - - - - - - - - - - - - - - - - - - - - - - 
\begin{frame}{I.2 Time Series Analysis (p4)}
\begin{itemize}
\item[] In order to perform forecasting or to understand the trends in the time series, we model the data using:
\item[]
\item \stronger{Moving Average (MA)}
\[
	\hat{x_{t+1}} = \frac{x_t + x_{t-1}+ \dots x_{t-N+1}}{N}
\]
\item \stronger{AutoRegressive (AR)}
\[
	\hat{x_{t+1}} = \beta_0 + \beta_1 x_{t-1}+\epsilon_t
\]
\item[] the above is AR(1)
\item[]
\item \stronger{Exponential Smoothing}
\item[] Exponentially decreases weights of older observations
\end{itemize}
\end{frame}
%- - - - - - - - - - - - - - - - - - - - - - - - - - - - - - - - - - - - - - - - 
%- - - - - - - - - - - - - - - - - - - - - - - - - - - - - - - - - - - - - - - - 
\begin{frame}{I.3. Time Series Mining (p1)}
\begin{itemize}
\item[] \stronger{Time Series Mining} applications can be segmented into:
\item[]
\item \strong{Representation and indexing}
\item \strong{Similarity measure}
\item \strong{Segmentation}
\item \strong{Visualisation}
\item \strong{Mining}
\item[] \cite{fu:2011}
\end{itemize}
\end{frame}
%- - - - - - - - - - - - - - - - - - - - - - - - - - - - - - - - - - - - - - - - 
\begin{frame}{I.3. Time Series Mining (p2)}
\begin{itemize}
\item[] \stronger{Representation and Indexing}
\item Time series representation is required for dimension reduction
\item Sampling is an option - a rate of $\frac{m}{n}$ is used where $m$ is the length of the time series and $n$ is the dimensionality after the sampling
\item Use the average value in a segment to represent the segment (piecewise aggregate approximation)
\item Linear interpolation - \strong{Piecewise Linear Representation (PLR)}
\item Linear regression
\item Reduce dimensionality but preserve points of interest (perceptually important points (PIP))
\item Transformations such as the \strong{Discrete Fourier Transform (DFT)}
\item[]
\item The efficiency of indexing depends only on the precision of the approximation in the reduced dimensionality space
\end{itemize}
\end{frame}
%- - - - - - - - - - - - - - - - - - - - - - - - - - - - - - - - - - - - - - - - 
\begin{frame}{I.3. Time Series Mining (p3)}
\begin{itemize}
\item[] \stronger{Similarity measure}
\item[]
\item Whole sequence matching - the length of all time series is considered during the similarity search
\item Subsequence matching where a query sequence $Q$ and a longer sequence $P$ are given and the task is to find the subsequences in $P$ that match $Q$
\item for example:
	\begin{itemize}
	\item Query 1: find all stocks that behave similar to stock A (whole sequence matching)
	\item Query 2: find all the patterns of a specific type in the past month of closing prices (subsequence matching)
	\end{itemize}
\item These can be done with \emph{Euclidean based measures} or other approaches such as \emph{dynamic time warping}
\item There are techniques specific to subsection matching
\end{itemize}
\end{frame}
%- - - - - - - - - - - - - - - - - - - - - - - - - - - - - - - - - - - - - - - - 
\begin{frame}{I.3. Time Series Mining (p4)}
\begin{itemize}
\item[] \stronger{Segmentation}
\item[]
\item Segmentation can be considered:
	\begin{itemize}
	\item A pre-processing step in preparation for a data mining task
	\item A trend analysis technique
	\end{itemize}
\item Simple discretisation can be achieved by using a fixed length window to segment series into subsequences, and then represented by the primitive patterns that are formed
\item Other approaches focus on detecting special events
\item Time points at which behaviour of time series change can be identified: change point detection 
\end{itemize}
\end{frame}
%- - - - - - - - - - - - - - - - - - - - - - - - - - - - - - - - - - - - - - - - 
\begin{frame}{I.3. Time Series Mining (p5)}
\begin{itemize}
\item[] Time series data mining tasks are the ultimate goal for either the original or the transformed data, common approaches are:
\item[]
\item \strong{Pattern discovery and clustering}
\item \strong{Classification}
\item \strong{Rule Mining}
\item \strong{Summarisation}
\end{itemize}
\end{frame}
%- - - - - - - - - - - - - - - - - - - - - - - - - - - - - - - - - - - - - - - - 
\begin{frame}{I.3. Time Series Mining (p6)}
\begin{itemize}
\item[] \stronger{Pattern Discovery}
\item[]
\item These tasks are also referred to as \strong{anomaly detection} or \strong{motif discovery}
\item A common group of techniques employed is \strong{distance-based clustering}
\item Depending on the size of the data these may require a compression of the time series
%\item ARMA and ARIMA have also been used as basis for clustering
\end{itemize}
\end{frame}
%- - - - - - - - - - - - - - - - - - - - - - - - - - - - - - - - - - - - - - - - 
\begin{frame}{I.3. Time Series Mining (p7)}
\begin{itemize}
\item[] \stronger{Classification}
\item[]
\item \strong{Classify} time series based on combining local properties or patterns in the time series
\item Another approach employs meta-features (recurring substructures) 
\item Many more approaches have been proposed....
\end{itemize}
\end{frame}
%- - - - - - - - - - - - - - - - - - - - - - - - - - - - - - - - - - - - - - - - 
\begin{frame}{I.3. Time Series Mining (p8)}
\begin{itemize}
\item[] \stronger{Rule discovery}
\item[]
\item Using \strong{association rule mining}
\item Discretize a time series into segments, convert each segment into a symbol and apply sequence analysis approaches
\item[]
\item[] Decision tree can also be employed for rule mining:
\item Sequential raw data is transformed into sequence of events
\item Temporal rules are inferred using classification trees 
\end{itemize}
\end{frame}
%- - - - - - - - - - - - - - - - - - - - - - - - - - - - - - - - - - - - - - - - 
%- - - - - - - - - - - - - - - - - - - - - - - - - - - - - - - - - - - - - - - - 
\begin{frame}{I. Summary: Time Series Data Mining}
\begin{itemize}
\item Target can be a time series - then objective is forecasting or decomposing the series
\item Time series can be additional data for a transaction, and be used to generate additional covariate attributes
\item Extraction of interesting patterns from the time series can enhance a data mining analysis
\item Time series can be the target to be classified or clustered
\item Rule mining can be applied to time series by reducing segments of the time series to symbolic representation
\end{itemize}
\end{frame}
%- - - - - - - - - - - - - - - - - - - - - - - - - - - - - - - - - - - - - - - - 
%- - - - - - - - - - - - - - - - - - - - - - - - - - - - - - - - - - - - - - - - 


%--------------------------------------------------------------------------------
\section{PART II}
%--------------------------------------------------------------------------------
\begin{frame}{Part II: Survival Data Mining}
\begin{itemize}
\item[II.1] Survival Analysis
\item[II.2] Survival Curves
\item[II.3] Hazard
\item[II.4] Censoring
\item[II.5] Forecasting
\item[II.6] Survival Trees
\end{itemize}
\end{frame}
%- - - - - - - - - - - - - - - - - - - - - - - - - - - - - - - - - - - - - - - - 
%- - - - - - - - - - - - - - - - - - - - - - - - - - - - - - - - - - - - - - - - 
\begin{frame}{II.1. Survival Analysis (p1)}
\begin{center}
\includegraphics[width=\textwidth]{images/dm_process_survival.pdf}
\end{center}
\end{frame}
%- - - - - - - - - - - - - - - - - - - - - - - - - - - - - - - - - - - - - - - - 
\begin{frame}{II.1. Survival Analysis (p2)}
\begin{itemize}
\item \stronger{Survival Analysis} is also known as \stronger{Time-to-Event Analysis}
\item[]
\item Traditionally linked to analysing clinical data
\item[]
\item It is also valuable in data mining for example:
\item[--] Understanding customer tenure
\item[--] Failure analysis in manufacturing
\item[--] Assessing risks across a customer's life cycle
\end{itemize}
\end{frame}
%- - - - - - - - - - - - - - - - - - - - - - - - - - - - - - - - - - - - - - - - 
\begin{frame}{II.1. Survival Analysis (p3)}
\begin{itemize}
\item[] \stronger{Survival analysis} in the context of typical \strong{data mining} applications can help answer the following questions:
\item[]
\item When a customer is likely to leave: \stronger{churn}
\item The next time a customer's behaviour will change
\item The next time a customer will change their contract
\item Factors that influence the likely customer \stronger{tenure}
\item Factors that significantly affect customer tenure
\end{itemize}
\end{frame}
%- - - - - - - - - - - - - - - - - - - - - - - - - - - - - - - - - - - - - - - - 
\begin{frame}{II.1. Survival Analysis (p4)}
\begin{itemize}
\item[] Terminologies
\item[]
\item \strong{Clinical} (medical)
	\begin{itemize}
	\item \strong{patient}
	\item \stronger{survival} - time elapsed between two events of interest
	\item \stronger{censoring} - the event is not experienced (i.e., the patient is still alive or disease-free)
	\end{itemize}
\item[]
\item \strong{Customer relationship} (marketing)
	\begin{itemize}
	\item \strong{customer} or \strong{contract}
	\item \strong{tenure} - time elapsed between start of contract and current time or end of contract
	\item \strong{censoring} - current customers will be censored (not considered) when applying survival analysis
	\end{itemize}
\end{itemize}
\end{frame}
%- - - - - - - - - - - - - - - - - - - - - - - - - - - - - - - - - - - - - - - - 
\begin{frame}{II.2. Survival Curves (p1)}
\begin{itemize}
\item Shows the proportion of customers/patients who are expected to survive up to a particular point in time (tenure)
\item Always starts at $100\%$  and then descends
\item May flatten out, but never increases
\item May descend all the way to zero ($0.0$)
\item Stopping (or leaving or dying) is a one-time event---
\item[] Once a customer or patient has experienced the event they cannot be re-included in the analysis
\end{itemize}
\end{frame}
%- - - - - - - - - - - - - - - - - - - - - - - - - - - - - - - - - - - - - - - - 
\begin{frame}{II.2. Survival Curves (p2)}
\begin{center}
\includegraphics[width=\textwidth]{images/lb3-figure10-1.png} \\
\cite[Figure 10.1]{LB3:2011}
\end{center}
\end{frame}
%- - - - - - - - - - - - - - - - - - - - - - - - - - - - - - - - - - - - - - - - 
\begin{frame}{II.2. Survival Curves (p3)}
\begin{itemize}
\item[] \stronger{Survival Function}
\item[]
\item Given data including an attribute $T$, where $T \geq 0$
\item[] where $T$ is the tenure or survival time
\item[]
\item The \stronger{survival function} is:
\[
	S(t) = Pr(T>t)
\]
\item $S(t)$ the probability that a subject will survive past time $t$
\item[]
\item At time $t=0$, $S(t)=1$
\item At time $t=\infty$, $S(t)=0$
\end{itemize}
\end{frame}
%- - - - - - - - - - - - - - - - - - - - - - - - - - - - - - - - - - - - - - - - 
\begin{frame}{II.2. Survival Curves (p4)}
\begin{itemize}
\item[] \stronger{Tenure} or \stronger{survival}
\item[]
\item In order to analyse customer \emph{survival} or \emph{tenure}, the following are required:
\item When a customer \strong{joins} or \strong{starts}
\item When a customer \strong{leaves} or \strong{stops}
\item[]
\item The difference between these two values is the customer \stronger{tenure}
\item[]
\item If a customer is still a customer at the time of ``analysis'', then they are \stronger{censored}
\item[]
\item If an attribute exists indicating the customer status (e.g., churned or is still a customer), then this can be used as the \strong{censoring indicator}
\end{itemize}  	
\end{frame}
%- - - - - - - - - - - - - - - - - - - - - - - - - - - - - - - - - - - - - - - - 
\begin{frame}{II.2. Survival Curves (p5)}
\begin{itemize}
\item[] Analysing the shape of the survival curve can help in comparing between groups:
\item[]
\item Proportion of customers who left at a determined time
\item[]
\item Time it takes for half the customers to leave:
\item[] \stronger{customer half life}
\item[] Also called \stronger{median customer lifetime}
\end{itemize}
\end{frame}
%- - - - - - - - - - - - - - - - - - - - - - - - - - - - - - - - - - - - - - - - 
\begin{frame}{II.3. Hazard (p1)}
\begin{itemize}
\item[] \stronger{Hazard} probability answers the following question:
\item[]
\item Assume a customer has a \strong{tenure} of $t$
\item \textbf{What is the probability that this customer leaves before $t+1$?}
\item[]
\item This can also be interpreted as the risk of losing customers between tenure $t$ and $t+1$
\end{itemize}
\end{frame}
%- - - - - - - - - - - - - - - - - - - - - - - - - - - - - - - - - - - - - - - - 
\begin{frame}{II.3. Hazard (p2)}
\begin{itemize}
\item A \stronger{Hazard curve} plots tenure (horizontal axis) versus probability (vertical axis), indicating the probability that a customer ends their tenure at a particular time 
\end{itemize}
\end{frame}
%- - - - - - - - - - - - - - - - - - - - - - - - - - - - - - - - - - - - - - - - 
\begin{frame}{II.3. Hazard (p3)}
\begin{itemize}
\item Example for a company that sells a subscription service:
\end{itemize}
\begin{center}
\includegraphics[width=\textwidth]{images/lb3-figure10-6.png}\\
\cite[Figure 10.6]{LB3:2011}
\end{center}
\end{frame}
%- - - - - - - - - - - - - - - - - - - - - - - - - - - - - - - - - - - - - - - - 
\begin{frame}{II.3. Hazard (p4)}
\begin{itemize}
\item Another example: life expectancy
\end{itemize}
\begin{center}
\includegraphics[width=\textwidth]{images/lb2-figure12-5.png} \\
\cite[Figure 12.5]{LB2:2004}
\end{center}
\end{frame}
%- - - - - - - - - - - - - - - - - - - - - - - - - - - - - - - - - - - - - - - - 
\begin{frame}{II.3. Hazard (p5)}
\begin{itemize}
\item Another example: comparison of customers who joined a service after being solicited by telephone vs email marketing 
\end{itemize}
\begin{center}
\includegraphics[width=\textwidth]{images/lb3-figure10-9.png} \\
\cite[Figure 10.9]{LB3:2011}
\end{center}
\end{frame}
%- - - - - - - - - - - - - - - - - - - - - - - - - - - - - - - - - - - - - - - - 
\begin{frame}{II.3. Hazard (p6)}
\begin{itemize}
\item \stronger{Hazard} is the probability that the customer stops at a particular tenure
\item[]
\item \stronger{Survival} is the probability of surviving until that tenure
\item[]
\item The survival curve is smoother and it is easier to use to calculate customer half life and tenure
\item[]
\item But as survival is cumulative, it can hide patterns occurring at specific times in the tenure
\end{itemize}
\end{frame}
%- - - - - - - - - - - - - - - - - - - - - - - - - - - - - - - - - - - - - - - - 
\begin{frame}{II.4. Censoring (p1)}
\begin{itemize}
\item Customers who have stopped and are no longer included in the count going forward

\item The \stronger{Kaplan-Meier (KM) Estimator} is a useful non-parametric tool for estimating survival functions
\item When there is \strong{no censoring} (i.e. all subjects included), then the survival function $S(t)$ is simply the proportion of survival times that are greater than $t$
\end{itemize}
\end{frame}
%- - - - - - - - - - - - - - - - - - - - - - - - - - - - - - - - - - - - - - - - 
\begin{frame}{II.4. Censoring (p2)}
\begin{itemize}
\item When there is \strong{censoring} (i.e. some subjects are not included), the survival times are ordered from smallest to largest:
\[
	\hat{S}(t)= \prod_{j|t_i \leq{t}} [1-\frac{d_j}{r_j}]
\]
\item[] where
\item[] $r_j$ is the number of patients at risk just before time $t_{(j)}$ 
\item[] $d_j$ is the number that experience the event of interest at $t_{(j)}$
\item Censored patients at time $t_{(j)}$ are included in $r_j$, so if a patient is lost to follow up at that time, they still count as at-risk until that time
\end{itemize}
\end{frame}
%- - - - - - - - - - - - - - - - - - - - - - - - - - - - - - - - - - - - - - - - 
\begin{frame}{II.4. Censoring (p3)}
\begin{itemize}
\item[] \strong{An Example}
\item Timeline legend:
	\begin{itemize}
	\item $\vdash$ = customer starts / joins
	\item $\circ$ = customer leaves
	\item $\bullet$ = customer is still active
	\end{itemize} 
\end{itemize}
\begin{center}
\begin{tabular}{cc}
\includegraphics[width=0.45\textwidth]{images/lb2-figure12-7.png} &
\includegraphics[width=0.55\textwidth]{images/lb2-table12-2.png} \\
Figure 12.7 & Table 12.2 \\
\multicolumn{2}{c}{\cite{LB2:2004}}\\
\end{tabular}
\end{center}
\end{frame}
%- - - - - - - - - - - - - - - - - - - - - - - - - - - - - - - - - - - - - - - - 
\begin{frame}{II.4. Censoring (p4)}
\begin{itemize}
\item[] Example (continued)
\item[]
\item Customers are either: ACTIVE, STOPPED or CENSORED
\end{itemize}
\begin{center}
\includegraphics[width=\textwidth]{images/lb2-table12-3.png} \\
\cite[Table 12.3]{LB2:2004}
\end{center}
\end{frame}
%- - - - - - - - - - - - - - - - - - - - - - - - - - - - - - - - - - - - - - - - 
\begin{frame}{II.4. Censoring (p5)}
\begin{itemize}
\item[] Example (continued)
\item[]
\item Here's the \stronger{hazard} calculation:
the number of customers who are STOPPED divided by the total of the customers who are either ACTIVE or STOPPED
\end{itemize}
\begin{center}
\includegraphics[width=0.8\textwidth]{images/lb2-table12-4.png} \\
\cite[Table 12.4]{LB2:2004}
\end{center}
\end{frame}
%- - - - - - - - - - - - - - - - - - - - - - - - - - - - - - - - - - - - - - - - 
\begin{frame}{II.5. Forecasting (p1)}
\begin{itemize}
\item Another way of exploiting time to event data is to use it for \strong{forecasting}, e.g., the number of future customers:
\item Model of existing customers using covariates
\item Apply the models to existing customers based on their current tenure - to produce ``stop forecast'' (i.e., when customer will churn)
\item Include new customers - to produce stop forecast
\item Produce an estimate of customer levels into the future
\end{itemize}
\end{frame}
%- - - - - - - - - - - - - - - - - - - - - - - - - - - - - - - - - - - - - - - - 
\begin{frame}{II.5. Forecasting (p2)}
\begin{itemize}
\item Example estimates (forecast) of when customers will stop
\end{itemize}
\begin{center}
\includegraphics[width=0.8\textwidth]{images/lb2-figure12-13.png} \\
\cite[Figure 12.13]{LB2:2004}
\end{center}
\end{frame}
%- - - - - - - - - - - - - - - - - - - - - - - - - - - - - - - - - - - - - - - - 
\begin{frame}{II.6. Survival Trees (p1)}
\begin{itemize}
\item \strong{Decision tree} and \strong{regression tree} algorithms have been used to model survival
\item At each node, the KM estimate of the survival function is reported
\item A variety of splitting criteria have been proposed:
\item[--] One method uses the \strong{logrank} statistic to compare the two groups composed of the nodes generated by a split~\cite{ciampi-et-al:1986}
\item[--] Leads to a split assuring best separation of the median survival times in the two resulting nodes 
\item[--] ``The logrank test is used to test the null hypothesis that there is no difference between the populations in the probability of an event at any time point.''~\cite{bland-altman:2004}
\end{itemize}
\end{frame}
%- - - - - - - - - - - - - - - - - - - - - - - - - - - - - - - - - - - - - - - - 
\begin{frame}{II.6. Survival Trees (p2)}
\begin{center}
\includegraphics[width=\textwidth]{images/survival-tree-example-trim.png}
\end{center}
\end{frame}
%- - - - - - - - - - - - - - - - - - - - - - - - - - - - - - - - - - - - - - - - 
%- - - - - - - - - - - - - - - - - - - - - - - - - - - - - - - - - - - - - - - - 
\begin{frame}{II. Summary: Survival Data Mining}
\begin{itemize}
\item A target attribute of type \emph{time to event} can be analysed using appropriate techniques
\item The presence of censoring has to be taken into account
\item The analysis can be done using traditional statistical approaches or in combination with data mining ones (e.g., survival trees)
\item The analysis of the \emph{time to event} attribute can be a step in a two-way process: estimate survival profiles then apply to current and new instances
\end{itemize}
\end{frame}
%- - - - - - - - - - - - - - - - - - - - - - - - - - - - - - - - - - - - - - - - 
%- - - - - - - - - - - - - - - - - - - - - - - - - - - - - - - - - - - - - - - - 


%--------------------------------------------------------------------------------
\section*{SUMMARY AND TO DO}
%--------------------------------------------------------------------------------
\begin{frame}{Summary: Time Series and Survival Data Mining}
\begin{itemize}
\item Time Series Data Mining: data mining where \emph{time} is a meaningful attribute 
\item Survival Data Mining: data mining where \emph{time to event} is a meaningful attribute
\end{itemize}
\end{frame}
%- - - - - - - - - - - - - - - - - - - - - - - - - - - - - - - - - - - - - - - - 
\begin{frame}{To Do}
\begin{enumerate}
\item READING:
	\begin{itemize}
	\item Chapter 12 from~\cite{LB2:2004} or Chapter 10 from~\cite{LB3:2011}
	\end{itemize}
\item Additional articles:
	\begin{itemize}
	\item \cite{chatfield:1975}
	\item \cite{fu:2011}
	\item \cite{kalbfleisch-prentice:1980}
	\item \cite{bou-hamad-et-al:2011}
	\end{itemize}
\item Assessment 1: Marks available 5th December

\end{enumerate}
\end{frame}
%- - - - - - - - - - - - - - - - - - - - - - - - - - - - - - - - - - - - - - - - 
%- - - - - - - - - - - - - - - - - - - - - - - - - - - - - - - - - - - - - - - - 



%--------------------------------------------------------------------------------
\section*{REFERENCES}
%--------------------------------------------------------------------------------
% use 'allowframebreaks' if refs flow onto multiple slides (each will be numbered 'REFERENCES I', 'REFERENCES II' etc)
\begin{frame}[allowframebreaks]{REFERENCES}
%\begin{frame}{REFERENCES}
\bibliographystyle{plain}
\bibliography{refs}
\end{frame}
%- - - - - - - - - - - - - - - - - - - - - - - - - - - - - - - - - - - - - - - - 
%- - - - - - - - - - - - - - - - - - - - - - - - - - - - - - - - - - - - - - - - 


\end{document}
